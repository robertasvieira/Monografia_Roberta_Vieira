\chapter{Introdução}
%\addcontentsline{toc}{chapter}{\Uppercase{Introdução}}
\label{introducao}

A Base Nacional Comum Curricular (BNCC), documento que define as diretrizes para a Educação Básica no Brasil, estabelece que a matemática deve ser ensinada desde a Educação Infantil até o Ensino Médio, devido à sua ampla aplicabilidade na sociedade atual e à sua potencialidade na formação de cidadãos críticos e conscientes (Brasil, 2018, p. 265). Com o intuito de garantir um ensino estruturado e sequencial, a BNCC organiza as habilidades matemáticas em cinco unidades temáticas: Números, Álgebra, Geometria, Grandezas e Medidas, e Probabilidade e Estatística.

Dentre essas unidades, a de Probabilidade e Estatística destaca-se por sua relevância, uma vez que permite aos alunos desenvolver habilidades para coletar, organizar, representar, interpretar e analisar dados em diversos contextos, o que é fundamental para realizar julgamentos bem fundamentados (Brasil, 2018, p. 274). Sob esse viés, o ensino de probabilidade na Educação Básica é indispensável para que os estudantes adquiram a capacidade de quantificar a incerteza associada a diferentes eventos, capacitando-os a tomar decisões mais informadas (Rocha, 2023).

Entretanto, apesar da importância do ensino de probabilidade, os resultados do Sistema de Avaliação da Educação Básica (SAEB) revelam um cenário preocupante. Na avaliação de matemática aplicada em 2021 nas turmas do Ensino Médio, apenas 28,4\% dos estudantes atingiram os níveis de proficiência 4 e 5,  que contemplam os conceitos de probabilidade (Brasil, 2021, p. 209). Esse desempenho insatisfatório não só reflete a aprendizagem insuficiente dos alunos nessa área, mas também aponta para uma eventual deficiência do modelo de ensino predominante. 

Conforme descrito por Oliveira (202), em diversas instituições educacionais prevalece o método de ensino no qual o professor é responsável por desenvolver integralmente o raciocínio matemático e fornecer modelos prontos para a resolução de exercícios. Consequentemente, os estudantes acabam memorizando os algoritmos, mas falham em compreender o significado dos resultados obtidos (Reis, 2022). Para consolidar efetivamente as habilidades matemáticas, é necessário integrar métodos complementares que incentivem a participação ativa dos alunos no processo de aprendizagem (Soares, 2020).

A utilização de recursos didáticos que atraiam a atenção dos alunos e apresentem novas perspectivas sobre a matemática, demonstrando que esta não se resume a regras e fórmulas pré-estabelecidas, pode transformar a sala de aula em um ambiente de aprendizagem mais significativo, segundo Rocha (2023). Entre os recursos disponíveis, os jogos educativos se destacam como ferramentas que auxiliam na compreensão e aplicação prática do conhecimento adquirido no ambiente escolar (Rocha, 2023). %%Adicione aqui sua questão norteadora.

Nesse contexto, o presente trabalho tem como objetivo geral desenvolver um jogo educativo digital com elementos de \textit{Role-Playing Game} (RPG), direcionado ao ensino de probabilidade no nível médio. Para atingir esse objetivo, foram delineados os seguintes objetivos específicos: 

        \begin{itemize}
        	
        	\item Selecionar os conceitos de probabilidade a serem abordados no jogo;
            \item Elaborar situações-problema que integrem os conceitos selecionados;
            \item Desenvolver uma narrativa imersiva que incorpore as situações-problema; 
            \item Identificar boas práticas para o desenvolvimento de jogos educativos; 
            \item Desenvolver o jogo educativo utilizando a \textit{engine} RPG Maker; 
            \item Aplicar o jogo em turmas do 3º ano do Ensino Médio; 
            \item Avaliar a eficácia do jogo desenvolvido.
            
        \end{itemize}
        
A utilização de um jogo educativo para o ensino de probabilidade justifica-se pela constatação de que o desempenho dos alunos tem sido insatisfatório com os métodos tradicionais, ao passo que estudos psicopedagógicos demonstram que o uso de jogos contribui significativamente com o desenvolvimento do raciocínio lógico-matemático e com a construção do pensamento crítico (Soares, 2020). Ademais, os jogos permitem que os alunos interajam com o conteúdo de forma mais envolvente e significativa, transformando a aprendizagem de uma mera obrigação em uma experiência atrativa (Reis, 2022).

A opção por um jogo digital fundamenta-se na necessidade de incorporar tecnologias digitais ao processo educativo. A BNCC ressalta que, ao considerar as experiências diárias dos estudantes do Ensino Médio, que são impactadas pelos avanços tecnológicos, é essencial utilizar tecnologias digitais e aplicativos como ferramentas para a investigação matemática (Brasil, 2018, p. 528). Além de refletir as realidades contemporâneas dos alunos, a integração de recursos tecnológicos no ensino de matemática pode ser uma ferramenta eficaz para atrair o interesse dos estudantes (Moura, 2020).

Este projeto de pesquisa está estruturado em cinco capítulos. O segundo capítulo expõe a revisão da literatura sobre o ensino de probabilidade, jogos educativos como ferramenta de aprendizagem e jogos de RPG no contexto educacional. O terceiro capítulo descreve os procedimentos metodológicos a serem empregados para a criação e implementação do jogo. No quarto capítulo, são apresentadas as expectativas em relação aos impactos do jogo educativo na aprendizagem dos alunos do Ensino Médio. Por fim, o quinto capítulo detalha a cronologia das atividades previstas para a execução do projeto.
